
%% anaka_paper.tex
%% V1.0
%% 2014/08/12
%% by Preethu Thomas
%%
%% First draft of the project to be submitted for EEL6935, Fall 2014
%% (requires IEEEtran.cls version 1.7 or later)
%%


\documentclass[12pt,journal,compsoc]{IEEEtran}
% The Computer Society requires 12pt.
% If IEEEtran.cls has not been installed into the LaTeX system files,
% manually specify the path to it like:
% \documentclass[10pt,journal,compsoc]{../sty/IEEEtran}


% *** CITATION PACKAGES ***
%
\ifCLASSOPTIONcompsoc
  % IEEE Computer Society needs nocompress option
  % requires cite.sty v4.0 or later (November 2003)
  % \usepackage[nocompress]{cite}
\else
  % normal IEEE
  % \usepackage{cite}
\fi
% cite.sty was written by Donald Arseneau
% V1.6 and later of IEEEtran pre-defines the format of the cite.sty package
% \cite{} output to follow that of IEEE. Loading the cite package will
% result in citation numbers being automatically sorted and properly
% "compressed/ranged". e.g., [1], [9], [2], [7], [5], [6] without using
% cite.sty will become [1], [2], [5]--[7], [9] using cite.sty. cite.sty's
% \cite will automatically add leading space, if needed. Use cite.sty's
% noadjust option (cite.sty V3.8 and later) if you want to turn this off.
% cite.sty is already installed on most LaTeX systems. Be sure and use
% version 4.0 (2003-05-27) and later if using hyperref.sty. cite.sty does
% not currently provide for hyperlinked citations.
% The latest version can be obtained at:
% http://www.ctan.org/tex-archive/macros/latex/contrib/cite/
% The documentation is contained in the cite.sty file itself.
%
% Note that some packages require special options to format as the Computer
% Society requires. In particular, Computer Society  papers do not use
% compressed citation ranges as is done in typical IEEE papers
% (e.g., [1]-[4]). Instead, they list every citation separately in order
% (e.g., [1], [2], [3], [4]). To get the latter we need to load the cite
% package with the nocompress option which is supported by cite.sty v4.0
% and later. Note also the use of a CLASSOPTION conditional provided by
% IEEEtran.cls V1.7 and later.

\usepackage{placeins}

% *** GRAPHICS RELATED PACKAGES ***
%
\ifCLASSINFOpdf
   \usepackage[pdftex]{graphicx}
  % declare the path(s) where your graphic files are
   \graphicspath{{../png/}}
  % and their extensions so you won't have to specify these with
  % every instance of \includegraphics
   \DeclareGraphicsExtensions{.pdf,.jpeg,.png}
\else
  % or other class option (dvipsone, dvipdf, if not using dvips). graphicx
  % will default to the driver specified in the system graphics.cfg if no
  % driver is specified.
   \usepackage[dvips]{graphicx}
  % declare the path(s) where your graphic files are
   \graphicspath{{../jpeg/}{../png/}}
  % and their extensions so you won't have to specify these with
  % every instance of \includegraphics
   \DeclareGraphicsExtensions{.eps}
\fi
% graphicx was written by David Carlisle and Sebastian Rahtz. It is
% required if you want graphics, photos, etc. graphicx.sty is already
% installed on most LaTeX systems. The latest version and documentation can
% be obtained at: 
% http://www.ctan.org/tex-archive/macros/latex/required/graphics/
% Another good source of documentation is "Using Imported Graphics in
% LaTeX2e" by Keith Reckdahl which can be found as epslatex.ps or
% epslatex.pdf at: http://www.ctan.org/tex-archive/info/
%
% latex, and pdflatex in dvi mode, support graphics in encapsulated
% postscript (.eps) format. pdflatex in pdf mode supports graphics
% in .pdf, .jpeg, .png and .mps (metapost) formats. Users should ensure
% that all non-photo figures use a vector format (.eps, .pdf, .mps) and
% not a bitmapped formats (.jpeg, .png). IEEE frowns on bitmapped formats
% which can result in "jaggedy"/blurry rendering of lines and letters as
% well as large increases in file sizes.
%
% You can find documentation about the pdfTeX application at:
% http://www.tug.org/applications/pdftex



%\usepackage{ps4pdf}
% dvi->ps workflow is required to use such packages as psfrag.sty and
% pstricks.sty. However, Rolf Niepraschk's ps4pdf.sty provides a way to
% apply psfrag/pstricks effects to .eps figures and then get the resultant
% figures in .pdf form. Thus, providing an easier way for migrating from
% .eps to .pdf figures. After ps4pdf.sty loads, if:
% 1. producing .dvi output: the output file will consist ONLY of the
%    figures (or other constructs encased within \PSforPDF commands)
% 2. producing .pdf output: pdflatex will look in the filename-pics.pdf
%    file, where filename is the basename of the tex document, for the
%    graphics (or other constructs encased within \PSforPDF commands).
%    NOTE: If you ever change your figures, you must remember to remake
%    the filename-pics.pdf file.
%
% This way you can do a:
% 
% latex filename
% dvips -Ppdf -o filename-pics.ps filename.dvi
% ps2pdf filename-pics.ps filename-pics.pdf
% 
% to produce a filename-pics.pdf graphics container that contains
% .pdf versions of the graphics with psfrag, pstricks, etc. features.
% Note that you will not typically be able to view the figures in 
% filename-pics.ps because of an offset. However, you will be able to
% view them in filename-pics.pdf. Also, note that when ps4pdf is in effect
% with .dvi output, you may get harmless over/under full box warnings - 
% ignore them. 
% Then, run pdflatex:
% 
% pdflatex filename
% 
% to use pdflatex to make PDF output, automatically using the figures in
% filename-pics.pdf. Alternatively, you could use dvips -i option to
% obtain separate .pdf files for each figure:
%
% dvips -Ppdf -i -E -o fig filename
%
% then convert each figure to pdf via a command such as epstopdf and then
% use pdflatex with these pdf figures and then to dispense with ps4pdf.
%
% Remember to rerun through latex/dvips/ps2pdf if you ever change your
% figures so that filename-pics.pdf gets updated.
% ps4pdf requires David Kastrup's preview-latex and a recent LaTeX system
% (circa 2001 or later). The ps4pdf package and documentation can be
% obtained at: http://www.ctan.org/tex-archive/macros/latex/contrib/ps4pdf/
% The preview-latex package and documentation can be obtained at:
% http://www.ctan.org/tex-archive/macros/latex/contrib/preview/
%
% provide a bogus \PSforPDF, even when not loading pd4pdf. This way we can
% stop loading ps4pdf.sty if we choose to make separate .pdf versions of
% each of our figures.
\providecommand{\PSforPDF}[1]{#1}
% Note that in order for ps4pdf to work, all commands related to psfrag,
% pstricks, etc. must be called within the PSforPDF command. This applies
% even when *loading* via \usepackage psfrag.sty, etc.


%\PSforPDF{\usepackage{psfrag}}
% psfrag.sty was written by Craig Barratt, Michael C. Grant, and
% David Carlisle. It allows you to substitute LaTeX commands for text in
% imported EPS graphic files. In this way, LaTeX symbols can be placed into
% graphics that have been generated by other applications. You must use
% latex->dvips->ps2pdf workflow (not direct pdf output from pdflatex) if
% you wish to use this capability because it works via some PostScript
% tricks. Alternatively, the graphics could be processed as separate files
% via psfrag and dvips, then converted to PDF for inclusion in the main file
% which uses pdflatex. ps4pdf.sty (above) provides a way of doing this all
% at once within the main file.
% Docs are in "The PSfrag System" by Michael C. Grant and David Carlisle.
% There is also some information about using psfrag in "Using Imported
% Graphics in LaTeX2e" by Keith Reckdahl which documents the graphicx
% package (see above). The psfrag package and documentation can be obtained
% at: http://www.ctan.org/tex-archive/macros/latex/contrib/psfrag/
% 
% Note that the current version of psfrag does not "turn itself off" when
% running under pdf output. This will result in a harmless warning
% about a non-PDF \special. However, to silence this, a bogus psfrag
% command can be provided instead of loading psfrag.sty when PDF output
% is being used. Thus, a more complex alternative conditional loading scheme
% can be employed instead of the straightforword way above:
%
%\ifCLASSINFOpdf
% if outputting PDF, do not use or load psfrag.sty as current versions
% output a non-PDF special that generates a harmless, but annoying warning.
% Instead, we provide a bogus \psfrag command that does nothing with
% its arguments. This is a tad tricky because \psfrag can have up to six
% arguments four of which are optional: \psfrag{}[][][][]{}
% Code based on that in psfrag.sty
%\makeatletter
%\def\psfrag{\@ifstar{\@BOGUSpsfraga}{\@BOGUSpsfraga}}
%\def\@BOGUSpsfraga{\begingroup
%   \@makeother\"\@makeother\*\@makeother\!\@makeother\~%
%   \@makeother\:\@makeother\\\@makeother\%\@makeother\#%
%   \@makeother\ \@BOGUSpsfragb}
%\def\@BOGUSpsfragb#1{\endgroup
%                \@ifnextchar [{\@BOGUSpsfragc}%
%                              {\@BOGUSpsfrag}}
%\def\@BOGUSpsfragc[#1]{\@ifnextchar [{\@BOGUSpsfragd}%
%                                     {\@BOGUSpsfrag}}
%\def\@BOGUSpsfragd[#1]{\@ifnextchar [{\@BOGUSpsfrage}%
%                                     {\@BOGUSpsfrag}}
%\def\@BOGUSpsfrage[#1]{\@ifnextchar [{\@BOGUSpsfragf}%
%                                     {\@BOGUSpsfrag}}
%\def\@BOGUSpsfragf[#1]{\@BOGUSpsfrag}
%\def\@BOGUSpsfrag#1{\ignorespaces}
%\makeatother
%\else
% using dvi output, load psfrag, but funnel it through PSforPDF
% as required by ps4pdf.sty
%\PSforPDF{\usepackage{psfrag}}
%\fi





% *** MATH PACKAGES ***
%
%\usepackage[cmex10]{amsmath}
% A popular package from the American Mathematical Society that provides
% many useful and powerful commands for dealing with mathematics. If using
% it, be sure to load this package with the cmex10 option to ensure that
% only type 1 fonts will utilized at all point sizes. Without this option,
% it is possible that some math symbols, particularly those within
% footnotes, will be rendered in bitmap form which will result in a
% document that can not be IEEE Xplore compliant!
%
% Also, note that the amsmath package sets \interdisplaylinepenalty to 10000
% thus preventing page breaks from occurring within multiline equations. Use:
%\interdisplaylinepenalty=2500
% after loading amsmath to restore such page breaks as IEEEtran.cls normally
% does. amsmath.sty is already installed on most LaTeX systems. The latest
% version and documentation can be obtained at:
% http://www.ctan.org/tex-archive/macros/latex/required/amslatex/math/





% *** SPECIALIZED LIST PACKAGES ***
%\usepackage{acronym}
% acronym.sty was written by Tobias Oetiker. This package provides tools for
% managing documents with large numbers of acronyms. (You don't *have* to
% use this package - unless you have a lot of acronyms, you may feel that
% such package management of them is bit of an overkill.)
% Do note that the acronym environment (which lists acronyms) will have a
% problem when used under IEEEtran.cls because acronym.sty relies on the
% description list environment - which IEEEtran.cls has customized for
% producing IEEE style lists. A workaround is to declared the longest
% label width via the IEEEtran.cls \IEEEiedlistdecl global control:
%
% \renewcommand{\IEEEiedlistdecl}{\IEEEsetlabelwidth{SONET}}
% \begin{acronym}
%
% \end{acronym}
% \renewcommand{\IEEEiedlistdecl}{\relax}% remember to reset \IEEEiedlistdecl
%
% instead of using the acronym environment's optional argument.
% The latest version and documentation can be obtained at:
% http://www.ctan.org/tex-archive/macros/latex/contrib/acronym/


\usepackage{algorithmic}
% algorithmic.sty was written by Peter Williams and Rogerio Brito.
% This package provides an algorithmic environment fo describing algorithms.
% You can use the algorithmic environment in-text or within a figure
% environment to provide for a floating algorithm. Do NOT use the algorithm
% floating environment provided by algorithm.sty (by the same authors) or
% algorithm2e.sty (by Christophe Fiorio) as IEEE does not use dedicated
% algorithm float types and packages that provide these will not provide
% correct IEEE style captions. The latest version and documentation of
% algorithmic.sty can be obtained at:
% http://www.ctan.org/tex-archive/macros/latex/contrib/algorithms/
% There is also a support site at:
% http://algorithms.berlios.de/index.html
% Also of interest may be the (relatively newer and more customizable)
% algorithmicx.sty package by Szasz Janos:
% http://www.ctan.org/tex-archive/macros/latex/contrib/algorithmicx/




% *** ALIGNMENT PACKAGES ***
%
%\usepackage{array}
% Frank Mittelbach's and David Carlisle's array.sty patches and improves
% the standard LaTeX2e array and tabular environments to provide better
% appearance and additional user controls. As the default LaTeX2e table
% generation code is lacking to the point of almost being broken with
% respect to the quality of the end results, all users are strongly
% advised to use an enhanced (at the very least that provided by array.sty)
% set of table tools. array.sty is already installed on most systems. The
% latest version and documentation can be obtained at:
% http://www.ctan.org/tex-archive/macros/latex/required/tools/


%\usepackage{mdwmath}
%\usepackage{mdwtab}
% Also highly recommended is Mark Wooding's extremely powerful MDW tools,
% especially mdwmath.sty and mdwtab.sty which are used to format equations
% and tables, respectively. The MDWtools set is already installed on most
% LaTeX systems. The lastest version and documentation is available at:
% http://www.ctan.org/tex-archive/macros/latex/contrib/mdwtools/


% IEEEtran contains the IEEEeqnarray family of commands that can be used to
% generate multiline equations as well as matrices, tables, etc., of high
% quality.


%\usepackage{eqparbox}
% Also of notable interest is Scott Pakin's eqparbox package for creating
% (automatically sized) equal width boxes - aka "natural width parboxes".
% Available at:
% http://www.ctan.org/tex-archive/macros/latex/contrib/eqparbox/





% *** SUBFIGURE PACKAGES ***
%\ifCLASSOPTIONcompsoc
%\usepackage[tight,normalsize,sf,SF]{subfigure}
%\else
%\usepackage[tight,footnotesize]{subfigure}
%\fi
% subfigure.sty was written by Steven Douglas Cochran. This package makes it
% easy to put subfigures in your figures. e.g., "Figure 1a and 1b". For IEEE
% work, it is a good idea to load it with the tight package option to reduce
% the amount of white space around the subfigures. Computer Society papers
% use a larger font and \sffamily font for their captions, hence the
% additional options needed under compsoc mode. subfigure.sty is already
% installed on most LaTeX systems. The latest version and documentation can
% be obtained at:
% http://www.ctan.org/tex-archive/obsolete/macros/latex/contrib/subfigure/
% subfigure.sty has been superceeded by subfig.sty.


%\ifCLASSOPTIONcompsoc
%  \usepackage[caption=false]{caption}
%  \usepackage[font=normalsize,labelfont=sf,textfont=sf]{subfig}
%\else
%  \usepackage[caption=false]{caption}
%  \usepackage[font=footnotesize]{subfig}
%\fi
% subfig.sty, also written by Steven Douglas Cochran, is the modern
% replacement for subfigure.sty. However, subfig.sty requires and
% automatically loads Axel Sommerfeldt's caption.sty which will override
% IEEEtran.cls handling of captions and this will result in nonIEEE style
% figure/table captions. To prevent this problem, be sure and preload
% caption.sty with its "caption=false" package option. This is will preserve
% IEEEtran.cls handing of captions. Version 1.3 (2005/06/28) and later 
% (recommended due to many improvements over 1.2) of subfig.sty supports
% the caption=false option directly:
%\ifCLASSOPTIONcompsoc
%  \usepackage[caption=false,font=normalsize,labelfont=sf,textfont=sf]{subfig}
%\else
%  \usepackage[caption=false,font=footnotesize]{subfig}
%\fi
%
% The latest version and documentation can be obtained at:
% http://www.ctan.org/tex-archive/macros/latex/contrib/subfig/
% The latest version and documentation of caption.sty can be obtained at:
% http://www.ctan.org/tex-archive/macros/latex/contrib/caption/




% *** FLOAT PACKAGES ***
%
%\usepackage{fixltx2e}
% fixltx2e, the successor to the earlier fix2col.sty, was written by
% Frank Mittelbach and David Carlisle. This package corrects a few problems
% in the LaTeX2e kernel, the most notable of which is that in current
% LaTeX2e releases, the ordering of single and double column floats is not
% guaranteed to be preserved. Thus, an unpatched LaTeX2e can allow a
% single column figure to be placed prior to an earlier double column
% figure. The latest version and documentation can be found at:
% http://www.ctan.org/tex-archive/macros/latex/base/


%\usepackage{stfloats}
% stfloats.sty was written by Sigitas Tolusis. This package gives LaTeX2e
% the ability to do double column floats at the bottom of the page as well
% as the top. (e.g., "\begin{figure*}[!b]" is not normally possible in
% LaTeX2e). It also provides a command:
%\fnbelowfloat
% to enable the placement of footnotes below bottom floats (the standard
% LaTeX2e kernel puts them above bottom floats). This is an invasive package
% which rewrites many portions of the LaTeX2e float routines. It may not work
% with other packages that modify the LaTeX2e float routines. The latest
% version and documentation can be obtained at:
% http://www.ctan.org/tex-archive/macros/latex/contrib/sttools/
% Documentation is contained in the stfloats.sty comments as well as in the
% presfull.pdf file. Do not use the stfloats baselinefloat ability as IEEE
% does not allow \baselineskip to stretch. Authors submitting work to the
% IEEE should note that IEEE rarely uses double column equations and
% that authors should try to avoid such use. Do not be tempted to use the
% cuted.sty or midfloat.sty packages (also by Sigitas Tolusis) as IEEE does
% not format its papers in such ways.


%\ifCLASSOPTIONcaptionsoff
%  \usepackage[nomarkers]{endfloat}
% \let\MYoriglatexcaption\caption
% \renewcommand{\caption}[2][\relax]{\MYoriglatexcaption[#2]{#2}}
%\fi
% endfloat.sty was written by James Darrell McCauley and Jeff Goldberg.
% This package may be useful when used in conjunction with IEEEtran.cls'
% captionsoff option. Some IEEE journals/societies require that submissions
% have lists of figures/tables at the end of the paper and that
% figures/tables without any captions are placed on a page by themselves at
% the end of the document. If needed, the draftcls IEEEtran class option or
% \CLASSINPUTbaselinestretch interface can be used to increase the line
% spacing as well. Be sure and use the nomarkers option of endfloat to
% prevent endfloat from "marking" where the figures would have been placed
% in the text. The two hack lines of code above are a slight modification of
% that suggested by in the endfloat docs (section 8.3.1) to ensure that
% the full captions always appear in the list of figures/tables - even if
% the user used the short optional argument of \caption[]{}.
% IEEE papers do not typically make use of \caption[]'s optional argument,
% so this should not be an issue. A similar trick can be used to disable
% captions of packages such as subfig.sty that lack options to turn off
% the subcaptions:
% For subfig.sty:
% \let\MYorigsubfloat\subfloat
% \renewcommand{\subfloat}[2][\relax]{\MYorigsubfloat[]{#2}}
% For subfigure.sty:
% \let\MYorigsubfigure\subfigure
% \renewcommand{\subfigure}[2][\relax]{\MYorigsubfigure[]{#2}}
% However, the above trick will not work if both optional arguments of
% the \subfloat/subfig command are used. Furthermore, there needs to be a
% description of each subfigure *somewhere* and endfloat does not add
% subfigure captions to its list of figures. Thus, the best approach is to
% avoid the use of subfigure captions (many IEEE journals avoid them anyway)
% and instead reference/explain all the subfigures within the main caption.
% The latest version of endfloat.sty and its documentation can obtained at:
% http://www.ctan.org/tex-archive/macros/latex/contrib/endfloat/
%
% The IEEEtran \ifCLASSOPTIONcaptionsoff conditional can also be used
% later in the document, say, to conditionally put the References on a 
% page by themselves.





% *** PDF, URL AND HYPERLINK PACKAGES ***
%
%\usepackage{url}
% url.sty was written by Donald Arseneau. It provides better support for
% handling and breaking URLs. url.sty is already installed on most LaTeX
% systems. The latest version can be obtained at:
% http://www.ctan.org/tex-archive/macros/latex/contrib/misc/
% Read the url.sty source comments for usage information. Basically,
% \url{my_url_here}.


% NOTE: PDF thumbnail features are not required in IEEE papers
%       and their use requires extra complexity and work.
%\ifCLASSINFOpdf
%  \usepackage[pdftex]{thumbpdf}
%\else
%  \usepackage[dvips]{thumbpdf}
%\fi
% thumbpdf.sty and its companion Perl utility were written by Heiko Oberdiek.
% It allows the user a way to produce PDF documents that contain fancy
% thumbnail images of each of the pages (which tools like acrobat reader can
% utilize). This is possible even when using dvi->ps->pdf workflow if the
% correct thumbpdf driver options are used. thumbpdf.sty incorporates the
% file containing the PDF thumbnail information (filename.tpm is used with
% dvips, filename.tpt is used with pdftex, where filename is the base name of
% your tex document) into the final ps or pdf output document. An external
% utility, the thumbpdf *Perl script* is needed to make these .tpm or .tpt
% thumbnail files from a .ps or .pdf version of the document (which obviously
% does not yet contain pdf thumbnails). Thus, one does a:
% 
% thumbpdf filename.pdf 
%
% to make a filename.tpt, and:
%
% thumbpdf --mode dvips filename.ps
%
% to make a filename.tpm which will then be loaded into the document by
% thumbpdf.sty the NEXT time the document is compiled (by pdflatex or
% latex->dvips->ps2pdf). Users must be careful to regenerate the .tpt and/or
% .tpm files if the main document changes and then to recompile the
% document to incorporate the revised thumbnails to ensure that thumbnails
% match the actual pages. It is easy to forget to do this!
% 
% Unix systems come with a Perl interpreter. However, MS Windows users
% will usually have to install a Perl interpreter so that the thumbpdf
% script can be run. The Ghostscript PS/PDF interpreter is also required.
% See the thumbpdf docs for details. The latest version and documentation
% can be obtained at.
% http://www.ctan.org/tex-archive/support/thumbpdf/
% Be sure and use only version 3.8 (2005/07/06) or later of thumbpdf as
% earlier versions will not work properly with recent versions of pdfTeX
% (1.20a and later).


% NOTE: PDF hyperlink and bookmark features are not required in IEEE
%       papers and their use requires extra complexity and work.
% *** IF USING HYPERREF BE SURE AND CHANGE THE EXAMPLE PDF ***
% *** TITLE/SUBJECT/AUTHOR/KEYWORDS INFO BELOW!!           ***
\newcommand\MYhyperrefoptions{bookmarks=true,bookmarksnumbered=true,
pdfpagemode={UseOutlines},plainpages=false,pdfpagelabels=true,
colorlinks=true,linkcolor={black},citecolor={black},pagecolor={black},
urlcolor={black},
pdftitle={Bare Demo of IEEEtran.cls for Computer Society Journals},%<!CHANGE!
pdfsubject={Typesetting},%<!CHANGE!
pdfauthor={Michael D. Shell},%<!CHANGE!
pdfkeywords={Computer Society, IEEEtran, journal, LaTeX, paper,
             template}}%<^!CHANGE!
%\ifCLASSINFOpdf
%\usepackage[\MYhyperrefoptions,pdftex]{hyperref}
%\else
%\usepackage[\MYhyperrefoptions,breaklinks=true,dvips]{hyperref}
%\usepackage{breakurl}
%\fi
% One significant drawback of using hyperref under DVI output is that the
% LaTeX compiler cannot break URLs across lines or pages as can be done
% under pdfLaTeX's PDF output via the hyperref pdftex driver. This is
% probably the single most important capability distinction between the
% DVI and PDF output. Perhaps surprisingly, all the other PDF features
% (PDF bookmarks, thumbnails, etc.) can be preserved in
% .tex->.dvi->.ps->.pdf workflow if the respective packages/scripts are
% loaded/invoked with the correct driver options (dvips, etc.). 
% As most IEEE papers use URLs sparingly (mainly in the references), this
% may not be as big an issue as with other publications.
%
% That said, recently Vilar Camara Neto introduced his breakurl.sty
% package which permits hyperref to easily break URLs even in dvi
% mode. Note that breakurl, unlike most other packages, must be loaded
% AFTER hyperref. The latest version of breakurl and its documentation can
% be obtained at:
% http://www.ctan.org/tex-archive/macros/latex/contrib/breakurl/
% breakurl.sty is not for use under pdflatex pdf mode. Versions 1.10 
% (September 23, 2005) and later are recommened to avoid bugs in earlier
% releases.
%
% The advanced features offer by hyperref.sty are not required for IEEE
% submission, so users should weigh these features against the added
% complexity of use. Users who wish to use hyperref *must* ensure that
% their hyperref version is 6.72u or later *and* IEEEtran.cls is version
% 1.6b or later.
% The package options above demonstrate how to enable PDF bookmarks
% (a type of table of contents viewable in Acrobat Reader) as well as
% PDF document information (title, subject, author and keywords) that is
% viewable in Acrobat reader's Document_Properties menu. PDF document
% information is also used extensively to automate the cataloging of PDF
% documents. The above set of options ensures that hyperlinks will not be
% colored in the text and thus will not be visible in the printed page,
% but will be active on "mouse over". USING COLORS OR OTHER HIGHLIGHTING
% OF HYPERLINKS CAN RESULT IN DOCUMENT REJECTION BY THE IEEE, especially if
% these appear on the "printed" page. IF IN DOUBT, ASK THE RELEVANT
% SUBMISSION EDITOR. You may need to add the option hypertexnames=false if
% you used duplicate equation numbers, etc., but this should not be needed
% in normal IEEE work.
% The latest version of hyperref and its documentation can be obtained at:
% http://www.ctan.org/tex-archive/macros/latex/contrib/hyperref/





% *** Do not adjust lengths that control margins, column widths, etc. ***
% *** Do not use packages that alter fonts (such as pslatex).         ***
% There should be no need to do such things with IEEEtran.cls V1.6 and later.
% (Unless specifically asked to do so by the journal or conference you plan
% to submit to, of course. )


% correct bad hyphenation here
\hyphenation{op-tical net-works semi-conduc-tor}


\begin{document}
%
% paper title
% can use linebreaks \\ within to get better formatting as desired
\title{Paralleled Fuzzy Time Series and Genetic Algorithm Prediction}
%
%
% author names and IEEE memberships
% note positions of commas and nonbreaking spaces ( ~ ) LaTeX will not break
% a structure at a ~ so this keeps an author's name from being broken across
% two lines.
% use \thanks{} to gain access to the first footnote area
% a separate \thanks must be used for each paragraph as LaTeX2e's \thanks
% was not built to handle multiple paragraphs
%
%
%\IEEEcompsocitemizethanks is a special \thanks that produces the bulleted
% lists the Computer Society journals use for "first footnote" author
% affiliations. Use \IEEEcompsocthanksitem which works much like \item
% for each affiliation group. When not in compsoc mode,
% \IEEEcompsocitemizethanks becomes like \thanks and
% \IEEEcompsocthanksitem becomes a line break with idention. This
% facilitates dual compilation, although admittedly the differences in the
% desired content of \author between the different types of papers makes a
% one-size-fits-all approach a daunting prospect. For instance, compsoc 
% journal papers have the author affiliations above the "Manuscript
% received ..."  text while in non-compsoc journals this is reversed. Sigh.

\author{Ankit Sirmorya,~\IEEEmembership{CISE,}
        Preethu Thomas,~\IEEEmembership{CISE,}
        \\Shantanu Kande,~\IEEEmembership{CISE,}
         and Sumeet Pande,~\IEEEmembership{CISE}% <-this % stops a space
%\IEEEcompsocitemizethanks{\IEEEcompsocthanksitem M. Shell is with the Department
%of Electrical and Computer Engineering, Georgia Institute of Technology, Atlanta,
%GA, 30332.\protect\\
% note need leading \protect in front of \\ to get a newline within \thanks as
% \\ is fragile and will error, could use \hfil\break instead.
%E-mail: see http://www.michaelshell.org/contact.html
%\IEEEcompsocthanksitem J. Doe and J. Doe are with Anonymous University.}% <-this % stops a space
%\thanks{Manuscript received April 19, 2005; revised January 11, 2007.}}
}

% note the % following the last \IEEEmembership and also \thanks - 
% these prevent an unwanted space from occurring between the last author name
% and the end of the author line. i.e., if you had this:
% 
% \author{....lastname \thanks{...} \thanks{...} }
%                     ^------------^------------^----Do not want these spaces!
%
% a space would be appended to the last name and could cause every name on that
% line to be shifted left slightly. This is one of those "LaTeX things". For
% instance, "\textbf{A} \textbf{B}" will typeset as "A B" not "AB". To get
% "AB" then you have to do: "\textbf{A}\textbf{B}"
% \thanks is no different in this regard, so shield the last } of each \thanks
% that ends a line with a % and do not let a space in before the next \thanks.
% Spaces after \IEEEmembership other than the last one are OK (and needed) as
% you are supposed to have spaces between the names. For what it is worth,
% this is a minor point as most people would not even notice if the said evil
% space somehow managed to creep in.

% The paper headers
\markboth{EEL6935: Cloud Computing and Storage,~Project Paper,~Fall 2014}%
{Sirmorya,~Pande,~Kande &~ Thomas}
% The only time the second header will appear is for the odd numbered pages
% after the title page when using the twoside option.
% 
% *** Note that you probably will NOT want to include the author's ***
% *** name in the headers of peer review papers.                   ***
% You can use \ifCLASSOPTIONpeerreview for conditional compilation here if
% you desire.



% The publisher's ID mark at the bottom of the page is less important with
% Computer Society journal papers as those publications place the marks
% outside of the main text columns and, therefore, unlike regular IEEE
% journals, the available text space is not reduced by their presence.
% If you want to put a publisher's ID mark on the page you can do it like
% this:
%\IEEEpubid{0000--0000/00\$00.00~\copyright~2007 IEEE}
% or like this to get the Computer Society new two part style.
%\IEEEpubid{\makebox[\columnwidth]{\hfill 0000--0000/00/\$00.00~\copyright~2007 IEEE}%
%\hspace{\columnsep}\makebox[\columnwidth]{Published by the IEEE Computer Society\hfill}}
% Remember, if you use this you must call \IEEEpubidadjcol in the second
% column for its text to clear the IEEEpubid mark (Computer Society jorunal
% papers don't need this extra clearance.)



% use for special paper notices
%\IEEEspecialpapernotice{(Invited Paper)}



% for Computer Society papers, we must declare the abstract and index terms
% PRIOR to the title within the \IEEEcompsoctitleabstractindextext IEEEtran
% command as these need to go into the title area created by \maketitle.
\IEEEcompsoctitleabstractindextext{%
\begin{abstract}
%\boldmath
Paralleled Fuzzy Time Series(FTS) and Genetic algorithm(GA) prediction intends to bring these algorithms together to create a prediction model over time based events, while leveraging the MapReduce paradigm to enable parallel execution. Fuzzy Time Series is a algorithm that allows prediction by developing a model based on historical time based events. Genetic Algorithm emulates natural selection, such that we can improve the forecasts of a given prediction model. This process is run till the prediction converges sufficiently with the training data.
\end{abstract}
% IEEEtran.cls defaults to using nonbold math in the Abstract.
% This preserves the distinction between vectors and scalars. However,
% if the journal you are submitting to favors bold math in the abstract,
% then you can use LaTeX's standard command \boldmath at the very start
% of the abstract to achieve this. Many IEEE journals frown on math
% in the abstract anyway. In particular, the Computer Society does
% not want either math or citations to appear in the abstract.

% Note that keywords are not normally used for peerreview papers.
\begin{IEEEkeywords}
Fuzzy Time Series, Genetic algorithm, MapReduce
\end{IEEEkeywords}}


% make the title area
\maketitle

\IEEEdisplaynotcompsoctitleabstractindextext

\IEEEpeerreviewmaketitle

\section{Introduction}
\IEEEPARstart{A}{ccurately} forecasting results for a given use case is always a challenge. It depends on multiple variables, there are exceptions which are not the norm that occur in everyday life and all related data for the given event may not be available.
\\\indent An algorithm can make ``as close to the truth'' prediction, if it has sufficient historical data and  it makes sense and relates the available data to the pattern of occurrences of that event. Weather forecasting, Election predictions, Stock market analysis and recommendations are few of the numerous examples that we see in everyday life, where prediction models are effectively used.
\\\indent In this project, we intend to use Fuzzy Time Series (FTS) \cite{FTS} and Genetic Algorithm (GA) \cite{GA}  prediction models and translate them into MapReduce paradigm for reliable, accurate and efficient forecasting of time based events.  
\\\indent The novelty of this implementation is to provide an end-to-end solution by bringing together FTS, GA and MapReduce. FTS and GA being used in conjunction should improve prediction results, while MapReduce should improve running time as well as increase the amount of data ingested.


\section{Related Work}
\\\indent Fuzzy Time Series (FTS) \cite{FTS} is a prediction model that has proved its ability to forecast in numerous use cases, such as tourism \cite{FTS-tourism}, enrollments \cite{FTS-enroll}, temperature \cite{FTS-temperature} and many more.
\\\indent Holland's Genetic algorithm \cite{GA} has also been successfully been used in multiple prediction examples \cite{GA-bankruptcy,GA-electric,GAEB} and also in conjunction with FTS \cite{FTS-GA}. Genetic algorithm has also been run as MapReduce to solve the OneMax problem \cite{GA-mapreduce}.

\section{Architecture}
 \indent For the purpose of implementation of our Hybrid Model we plan to predict the time-based events. As an input to our prediction model we have used the data pertaining to number of events per year for the previous years as our statistical data input.
\\\indent The architecture is divided into three modules data ingestion, prediction and data visualization. This is as shown in \emph{Figure~\ref{arch}}. 

\begin{figure}
\includegraphics[width=2.25in]{arch}
\caption{Architecture of the implementation}
\label{arch}
\end{figure}

\subsection{Data Ingestion}
This phase will allow a configurable way of parsing the input data. Initially data will be read from a statistical data source such as publicly available data sets relevant to our prediction model. We have processed the data by application of MapReduce Paradigm so as to fine tune the data as per the input specification of GA+FTS Hybrid Prediction Model.

\subsection{Prediction}
The prediction module involves implementation of GA+FTS hybrid model on the incoming data from our data set. The subsystems for this model are explained below:
 
\subsubsection{Genetic Algorithm (GA)}
 GA mimics the evolutionary principle of “Survival of the Fittest”. GA can be defined as a set of computational methods that are inspired by evolutionary biology such as mutation, selection, inheritance and crossover that can be used to find solutions to optimization problem. The first step of GA is to generate a random population of possible solution which correspond to a single generation. After this fitness value of every solution is calculated and multiple solutions are selected on basis of their fitness levels to generate a new generation. This process is carried on iteratively till a point of convergence is reached where there is no significant improvement in a generation over previous generation. 

\subsubsection{ Fuzzy Time Series (FTS)}
 In our prediction model FTS is basically used as a fitness calculator that is to calculate correctness of prediction for each individual solution as a function of Root Mean Square Error (MSE).This is done by dividing the input data for time bound event into even and equal lengths of interval and then random selection of individuals that are product of GA iteration.
 
Thus the working of our hybrid GA+FTS prediction model can be visualized as combination of two functionalities where GA is used to improve correctness of each generation over previous generations and FTS is used as a fitness calculator to check the correctness of prediction in terms of mean square error.
 

\subsection{Data Visualization}
The result data visualization phase will use the prediction model results and display them in various output formats in order to easily understand and visualize the results. 

\section{Design Steps}
\subsection{Algorithm}


The algorithm for this project: 
\\\indent \textbf{Step 1:} Read the training data which in this case will be time based events.
\\\indent \textbf{Step 2:} Find the maximum and the minimum values of the training data and compute the Universe of Discourse.
\\\indent \textbf{Step 3:} Create the first generation of Genetic Algorithm which will be a set of randomized individuals. Each of the individuals will be within the Universe of Discourse.
\\\indent \textbf{Step 4:} Create the division of the generation data which can be used in the MapReduce model. 
\\\indent \textbf{Step 5:} The mapper task will apply Fuzzy Time Series Algorithm using the training data and GA population. This will make prediction for each individual and fitness value for each of them will be calculated.
\\\indent \textbf{Step 6:} The reducer task will perform the selection, cross-over and mutation operations of GA based on the fitness value calculated by the mapper task. Thus, a new generation of individuals will be created.
\\\indent \textbf{Step 7:} Repeat Steps 4 through 6 until the convergence point is not reached.

\subsection{Flowchart}
\emph{Figure~\ref{flowchart}} is the flowchart for this implementation
\begin{figure}
\includegraphics[width=2.5in]{flowchart}
\caption{Flowchart}
\label{flowchart}
\end{figure}

\section{Modules}
\indent There are three major modules namely data ingestion, prediction and data visualization.

\subsection{Data ingestion}
\emph{Figure~\ref{dp_interact}} shows the interaction diagram of the Data Ingestion Module.

\begin{figure*}
\includegraphics[height=2.5in,width=6.5in]{dp_interact}
\caption{Data Ingestion Interaction Diagram}
\label{dp_interact}
\end{figure*}
\indent Data ingestion is used to consume raw data and produce output which will serve as the input to the prediction module. We have used MapReduce model to parse and raw data and generated the desired input. In this module we have collected time based data which is readily available. For example, the data format will be month wise data for the events for specific region starting from year 1971. The data parser module will use MapReduce paradigm to giveout as year wise format along with the minimum and maximum value from same.
\\\indent The MapReduce framework operates on \textless key, value\textgreater  pairs, that is, the framework accepts the input to the job as a set of \textless key, value\textgreater  pairs and produces a set of \textless key, value\textgreater  pairs as the output of the job, in different types. The \textless  key, value \textgreater  pair here will be \textless year, no. of events\textgreater . The function of Maps are to transform individual tasks that transform input records into intermediate records. Output of data maps, that is transformed intermediate records do not need to be of the same type as the input records. Reducer reduces a set of intermediate values which share a key to a smaller set of values. Sorted output of the mappers will be the input to reducer. The output of the reducer will be the sum of all events in a particular year.

\subsection{Prediction}
\indent Figure~\ref{gafts_class_1}, Figure~\ref{gafts_class_2}, Figure~\ref{gafts_class_3}, and Figure~\ref{gafts_interact} are the class and interaction diagrams of the Prediction module.


\begin{figure}
\includegraphics[height=6in,width=3in]{gafts_class_1}
\caption{Prediction module Class diagram - 1}
\label{gafts_class_1}
\end{figure}
\begin{figure}
\includegraphics[height=5.5in,width=3in]{gafts_class_2}
\caption{Prediction module Class diagram - 2}
\label{gafts_class_2}
\end{figure}
\begin{figure}
\includegraphics[height=2in,width=3in]{gafts_class_3}
\caption{Prediction module Class diagram - 3}
\label{gafts_class_3}
\end{figure}

\begin{figure*}
\includegraphics[height=2.5in,width=6.5in]{gafts_interact}
\caption{Prediction module Interaction diagram}
\label{gafts_interact}
\end{figure*}


\\\indent Prediction module will include implementation of FTS and GA algorithms in the MapReduce paradigm. With the output from  the data ingestion module, the minimum and maximum value as 440 and 1067. The universe of discourse U = [Dmin – D1, Dmax + D2], where Dmin and Dmax denote the minimum events and the maximum events at particular region shown in \emph{Figure~\ref{parseddata}}, and D1 and D2 are two proper values. 
\\\indent The universe of discourse U will be divided into several even and equal length intervals. Here we have selected D1 as 40 and D2 as 33. Then we will generate population between threshold values which will be our first generation.

\begin{figure}
\includegraphics[height=3in]{parseddata}
\caption{Output of Data Ingestion}
\label{parseddata}
\end{figure}
	 	 	
\\\indent FTS is used as a fitness calculator which calculates the correctness of prediction for each individual as a function of Mean Square Error (MSE). FTS fuzzifies the historical data of the time-based events and generates fuzzy logical relationships based on the order. GA will be used to improve correctness for each generation over the previous generation by making use of GA operators which use the calculated fitness values. The calculated MSE value of each derived chromosome obtained in the new generation, and the same cycle is again implemented till a point of convergence is reached. In this case, the fittest individual will be the one having the lowest mean square error. If the minimum mean square error doesn’t change over subsequent generations then the convergence point of the Genetic Algorithm is reached. \emph{Fig. 5.} is the output of mean squared error between predicted and actual values. 

\begin{figure}
\includegraphics[width=2.5in]{genop}
\caption{Output of Mean Square Error over generations}
\end{figure}

\subsection{Data Visualization}

There are many different types of tools both web based and desktop based which will be very helpful for data visualization. We are exploring the use of Chart.js, Play framework and https://plot.ly/ (web based tool) to showcase our results.
 \\\indent Chart.js uses the HTML5 canvas element. Chart.js is supported in all modern web browsers, it is javascript based and is lightweight. Play framework is web application framework built using Scala. It is asynchronous and lightweight.
 \\\indent Plotly is web based tool in which we can upload data to create and style charts with Plotly's online spreadsheet. Using  Plotly \emph{Figure~\ref{datav}} is generated, which is the reduction of MSE over generations.

\begin{figure}
\includegraphics[width=3.5in]{datav}
\caption{Reducing Mean Squared Error over the generation}
\label{datav}
\end{figure}

\section{Results	}
The two main metrics for testing our forecasting models will be accuracy and running time. Higher accuracy would be indicated by comparing forecasted results with historical data, that is a lower Mean Squared Error between predicted and actual data sets. \\\indent Another result that we intend to achieve is significant drop in running time by implementing the same in distributed/parallel (Hadoop Framework) manner over normal sequential manner. 
\begin{figure}
\includegraphics[width=3.6in]{speedup}
\caption{Mock up of Speedup using MapReduce}
\end{figure}

\emph{Fig. 6.} shows the expected differences in running time when the GA+FTS module is used in sequential manner and the same model being used by making use of MapReduce Paradigm. The results were generated on an Intel Core i3 @1.90GHz X4, 64 bits, 4GB RAM.
We have observed some trends which are as follows, as number of intervals in the universe of discourse are increased, the MSE tends to decrease and as the order of fuzzy logical relationships increases, the MSE will decrease.

\begin{figure}
\includegraphics[width=3.65in]{predvsact}
\caption{Actual versus Mock Predicted values}
\end{figure}



\section{Novelty of the Idea}
Though there have been previous researches done in the field of parallel evolutionary algorithm \cite{GA-parallel}, in this paper we have tried to overcome the time incurred by GA-FTS prediction model. 
\\\indent Also, with the advent of MapReduce \cite{mapreduce} paradigm, it is much easier to abstract parallel system operations; while paving the way to concentrate on the problem at hand. However, a restrictive programming model like MapReduce, requires the translation of the algorithm in the form of map/reduce functions. This is still a major challenge and needs to be done with much consideration. 

\section{Possibilities  to Consider}
\indent During the execution of the project, the following will still be in consideration:

  \\\indent \textbf{Can we dynamically retrieve data?} Initially, we will be manually downloading/creating the data.
  \\\indent \textbf{Can we use some other parallel framework?} Possible candidates could be Spark \cite{spark-site}, Storm \cite{storm-site}, Giraph \cite{giraph-site} or Hama \cite{hama-site}
  \\\indent \textbf{Can we employ different MapReduce models of GA?} Initially, only selection, crossover and mutation stages will be in reducer phase. We need to evaluate different mechanisms to improve the reduce phase.
  \\\indent \textbf{Can we dynamically set thresholds for FTS/GA and number of individuals in GA?} We will be manually tweaking these threshold values and visualizing the effects. We will try to have a dynamic way to sample training data and running multiple instance of the algorithm in parallel to decide appropriate threshold values.


% Note that IEEE typically puts floats only at the top, even when this
% results in a large percentage of a column being occupied by floats.
% However, the Computer Society has been known to put floats at the bottom.


% An example of a double column floating figure using two subfigures.
% (The subfig.sty package must be loaded for this to work.)
% The subfigure \label commands are set within each subfloat command, the
% \label for the overall figure must come after \caption.
% \hfil must be used as a separator to get equal spacing.
% The subfigure.sty package works much the same way, except \subfigure is
% used instead of \subfloat.
%
%\begin{figure*}[!t]
%\centerline{\subfloat[Case I]\includegraphics[width=2.5in]{subfigcase1}%
%\label{fig_first_case}}
%\hfil
%\subfloat[Case II]{\includegraphics[width=2.5in]{subfigcase2}%
%\label{fig_second_case}}}
%\caption{Simulation results}
%\label{fig_sim}
%\end{figure*}
%
% Note that often IEEE papers with subfigures do not employ subfigure
% captions (using the optional argument to \subfloat), but instead will
% reference/describe all of them (a), (b), etc., within the main caption.


% An example of a floating table. Note that, for IEEE style tables, the 
% \caption command should come BEFORE the table. Table text will default to
% \footnotesize as IEEE normally uses this smaller font for tables.
% The \label must come after \caption as always.
%
%\begin{table}[!t]
%% increase table row spacing, adjust to taste
%\renewcommand{\arraystretch}{1.3}
% if using array.sty, it might be a good idea to tweak the value of
% \extrarowheight as needed to properly center the text within the cells
%\caption{An Example of a Table}
%\label{table_example}
%\centering
%% Some packages, such as MDW tools, offer better commands for making tables
%% than the plain LaTeX2e tabular which is used here.
%\begin{tabular}{|c||c|}
%\hline
%One & Two\\
%\hline
%Three & Four\\
%\hline
%\end{tabular}
%\end{table}


% Note that IEEE does not put floats in the very first column - or typically
% anywhere on the first page for that matter. Also, in-text middle ("here")
% positioning is not used. Most IEEE journals use top floats exclusively.
% However, Computer Society journals sometimes do use bottom floats - bear
% this in mind when choosing appropriate optional arguments for the
% figure/table environments.
% Note that, LaTeX2e, unlike IEEE journals, places footnotes above bottom
% floats. This can be corrected via the \fnbelowfloat command of the
% stfloats package.




% if have a single appendix:
%\appendix[Proof of the Zonklar Equations]
% or
%\appendix  % for no appendix heading
% do not use \section anymore after \appendix, only \section*
% is possibly needed

% use appendices with more than one appendix
% then use \section to start each appendix
% you must declare a \section before using any
% \subsection or using \label (\appendices by itself
% starts a section numbered zero.)
%



\begin{thebibliography}{1}
\bibitem{FTS}
Q. Song, B. Chissom, \emph{Fuzzy time series and its models}, Fuzzy Sets and Systems, Vol 54, 1993, p.~269-277.
\bibitem{FTS-tourism}
C.H, Wang and L.C. Hsu, \emph{Constructing and applying an improved fuzzy time series model: Taking the tourism industry for example}, Expert Systems with Applications, Vol 34, 2008, p.~2732-2738.
\bibitem{FTS-enroll}
S.M. Chen, \emph{Forecasting enrollments based on high-order fuzzy time series}, Cybernetics and Systems: An International Journal, Vol 33, 2002, p.~1-16.~781-789.
\bibitem{FTS-temperature}
S.M. Chen and J.R. Hwang, \emph{Temperature prediction using fuzzy time series}, IEEE Transactions on Systems, Man, and Cybernetics, Part B: Cybernetics, Vol 30, Issue 2, 2000, p.~263-275.
\bibitem{GA}
John H. Holland, \emph{Adaptation in Natural and Artificial Systems}, 1975, University of Michigan Press, Ann Arbor.
\bibitem{GA-bankruptcy}
K.S. Shin and Y.J. Lee, \emph{A genetic algorithm application in bankruptcy prediction modeling}, Expert Systems with Applications, 2002,~321-328
\bibitem{GA-electric}
A. Azadeh, S.F. Ghaderi, S. Tarverdian and M. Saberi. \emph{Integration of artificial neural networks and genetic algorithm to predict electrical energy consumption}, Applied Mathematics and Computation, 2007,~186:1731-41
\bibitem{FTS-GA}
S.M. Chen and N.Y. Chung, \emph{Forecasting enrollments using high-order fuzzy time series and genetic algorithms}, Internat. J. Intell. Syst. 21 ,2006,~485-501.
\bibitem{mapreduce}
Jeffrey Dean and Sanjay Ghemawat, \emph{MapReduce:~Simplified Data Processing on Large Clusters},OSDI'04: Sixth Symposium on Operating System Design and 
Implementation,~107-113.
\bibitem{GA-mapreduce}
A. Verma, X. Llora, D.E. Goldberg, and R.H.Campbell, \emph{Scaling genetic algorithms using mapreduce}, Intelligent Systems Design and Applications, 2009,~13-18.
\bibitem{GAEB}
Deepak Singh and Ankit Sirmorya, \emph{Article: Solving Real Optimization Problem Using Genetic Algorithm with Employed Bee (GAEB)}, International Journal of Computer Applications, 2012, 1-5.
\bibitem{GA-parallel}
E. Cantú-Paz, \emph{A survey of parallel genetic algorithms}, Calculateurs paralleles, reseaux et systems repartis, 1998,~141-171.
\bibitem{spark-site}
Apache Spark - \emph{https://spark.apache.org/}
\bibitem{storm-site}
Apache Storm - \emph{https://storm.incubator.apache.org/}
\bibitem{giraph-site}
Apache Giraph - \emph{http://giraph.apache.org/}
\bibitem{hama-site}
Apache Hama - \emph{https://hama.apache.org/}
\end{thebibliography}

% You can push biographies down or up by placing
% a \vfill before or after them. The appropriate
% use of \vfill depends on what kind of text is
% on the last page and whether or not the columns
% are being equalized.

%\vfill
% Can be used to pull up biographies so that the bottom of the last one
% is flush with the other column.
%\enlargethispage{-5in}



% that's all folks
\end{document}